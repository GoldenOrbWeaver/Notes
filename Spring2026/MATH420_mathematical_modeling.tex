\documentclass{article}
\usepackage{graphicx} % Required for inserting images
\usepackage{tikz}
\usepackage{physics}
\usepackage{gensymb}
\usepackage{amsmath, amsfonts, mathtools, amsthm, amssymb, mathrsfs}
\usepackage{titlesec}
\usepackage{fancyhdr}
\usepackage{hyperref}
\usepackage{float}
\usepackage{booktabs}
\usepackage{enumitem}
\usepackage{subcaption}
\usepackage{multicol}
\usepackage{import}
\usepackage[usenames,dvipsnames]{xcolor}
\usepackage[a4paper, margin=1in]{geometry}
% The configuration for boxes and parts of the overall idea come from Pingbang Hu: https://github.com/sleepymalc
% Use a relative path (forward slashes or "..") to avoid backslash-escapes
\subimport{../}{header.tex}

\title{\textbf{MATH 420 Mathematical Modeling}}
\author{GoldenOrbWeaver}
\date{Spring 2026}

\begin{document}

\maketitle

\tableofcontents{\textbf{}}

\pagebreak

\pagestyle{fancy}
\fancyhead{}
\fancyhead[L]{\thetitle}
\fancyhead[R]{GoldenOrbWeaver}

\section{Ch.1 One Variable Optimization}
Approach to mathematical modeling:
1. Ask the question
2. Select the modeling approach
3. Formulate the model
4. Solve the problem
5. Answer the question
(possibly move later, reformat)

\begin{eg}
    A pig weighing 200 pounds gains 5 pounds per day and costs 45 cents a day 
    to keep. The market price for pigs is 65 cents per pound, but is falling
     1 center per day. When should the pig be sold?
\end{eg}

\begin{answer}
    Start at \(t=0\), with \(t\)  in days \\
    Weight of pig: \(w(t)=200+5t\)  \\
    Cost of pig: \[c(t)=0.45t\] \\
    Price per pound to sell pig: \[p(t)=0.65-0.01t\] \\
    Revenue from selling pig: \[r(t)=w(t)p(t)\]\\
    Profit from selling at time t: \[p(t)=r(t)-c(t)\]
\end{answer}


Look into math modeling competitions (SIAM, consortium)
download and learn julia

\end{document}