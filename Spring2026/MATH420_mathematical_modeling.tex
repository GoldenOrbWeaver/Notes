\documentclass{article}
\usepackage{graphicx} % Required for inserting images
\usepackage{tikz}
\usepackage{physics}
\usepackage{gensymb}
\usepackage{amsmath, amsfonts, mathtools, amsthm, amssymb, mathrsfs}
\usepackage{titlesec}
\usepackage{fancyhdr}
\usepackage{hyperref}
\usepackage{float}
\usepackage{booktabs}
\usepackage{enumitem}
\usepackage{subcaption}
\usepackage{multicol}
\usepackage{import}
\usepackage[usenames,dvipsnames]{xcolor}
\usepackage[a4paper, margin=1in]{geometry}
% The configuration for boxes and parts of the overall idea come from Pingbang Hu: https://github.com/sleepymalc
% Use a relative path (forward slashes or "..") to avoid backslash-escapes
\subimport{../}{header.tex}

\title{\textbf{MATH 420 Mathematical Modeling}}
\author{GoldenOrbWeaver}
\date{Spring 2026}


\begin{document}

\maketitle

\tableofcontents{\textbf{}}

\pagebreak

\pagestyle{fancy}
\fancyhead{}
\fancyhead[L]{\thetitle}
\fancyhead[R]{GoldenOrbWeaver}

\section{One Variable Optimization}
Approach to mathematical modeling:
\begin{enumerate}
    \item Ask the question
    \item Select the modeling approach
    \item Formulate the model
    \item Solve the problem
    \item Answer the question
\end{enumerate}

\begin{eg}
    A pig weighing 200 pounds gains 5 pounds per day and costs 45 cents a day 
    to keep. The market price for pigs is 65 cents per pound, but is falling
     1 center per day. When should the pig be sold?
\end{eg}

\begin{answer}
    Start at \(t=0\), with \(t\)  in days. Then we find each component:\\
    Weight of pig:   \\
    \[
        w(t)=200+5t
    \]
    Cost of pig: \[c(t)=0.45t\] \\
    Price per pound to sell pig: \[p(t)=0.65-0.01t\] \\
    Revenue from selling pig: \[r(t)=w(t)p(t)\]\\
    Profit from selling at time t: \[p(t)=r(t)-c(t)\]
\end{answer}

However, we aren't sure that our values are accurate. The amount the price
falls per day could be different, so we can instead set it to a function \(r(t)\)
then:
\[
    p(t)=0.65-rt
\]

To find the optimal time to sell, solve \(p^{\prime} (t)=0\) to get \(t_{max}\). 
We are interested in how sensitive \(t_{max}\) is on changes in \(r\), which
we call \(s(t,r)\):
\[
    s(t,v)=\lim{\frac{\frac{\Delta t}{t}}{\frac{\Delta r}{r}}}=\frac{r}{t}*\frac{dt}{dr}
\]
We begin to solve:
\[
    p^{\prime} (t)=r^{\prime} (t)-c^{\prime} (t) \\
    =w^{\prime} (t)p^{\prime} (t)=w(t)p^{\prime} (t)-0.45 \\
    =5(0.65-rt)+(200+5t)(-r)-0.45 \\
    =-10rt-200r+2.80=0
\]
Then, we solve for \(t_{max}\):
\[
    t_{max}=\frac{2.80-200r}{10r}=0.28r^{-1}-20
\] 
Next, we find \(\frac{dt_{max}}{dr}\):
\[
    \frac{dt_{max}}{dr}=-0.28r^{-2}
\] 
Therefore:
\[
    S(t,r)=\frac{-r}{t_{max}}*\frac{0.28}{r^2}=\frac{-0.28}{t_{max}r}
\]
Substituting \(t_{max}\):
\[
    S(t,r)=\frac{-0.28}{(0.28r^{-1}-20)r}=\frac{-0.28}{0.28-20r}
\] 
At our best guess of \(r=0.01\):
\[
    S(t,0.01)=\frac{-0.28}{0.28-0.2}=\frac{-0.28}{0.08}=\frac{-7}{2}
\] 
Thus, if \(r\) is wrong by \(\pm10\%\), then \(t_{max}\) will be wrong by \(S(t,r)*10\%=\frac{7}{2}*10\%=\pm35\%\) 

What if the rate of growth of the pig is also wrong? We redefine \(w(t)\):
\[
    w(t)=200+gt
\] 
Then, we find \(S(t,g)\):
\[
    S(t,g)=\frac{g}{t_{max}}*\frac{dt_{max}}{dg}|_{g=5}\approx3.0625
\] 
We can also examine the sensitivity of the outcome:
\[
    S(P,r)=\frac{r}{P}*\frac{dP}{dr}=\frac{r}{P(t_{max})}*\frac{dP(t)}{dt}*\frac{dt}{dr}|_{t=8}=0
\]
This means that the profit doesn't change that much (to the first order) when the rate of change in the price of the pig varies.

\textbf{(FIX LAST PART WITH dp/dr, on website. ALSO NEW PLOT)}

\begin{eg}
    A manufacturer of color TV sets is planning the introduction of two new products,
    a 19-inch LCD flat panel set with a manufacturer's suggested retail price 
    (MSRP) of \$339 and a 21-inch LCD flat panel set with an MSRP of \$399. The cost to
    the company is \$195 per 19-inch set and \$225 per 21-inch set, plus an additional 
    \$400,000 in fixed costs. In the competitive market in which these sets 
    will be sold, the number of sales per year will affect the average selling 
    price. It is estimated that for each type of set, the average selling price drops by
    one cent for each additional unit sold. Furthermore, sales of the 19-inch set will affect
    sales of the 21-inch set and vice-versa. It is estimated that the average selling 
    price for the 19-inch set will be reduced by an additional 0.3 cents for each
    21-inch set sold, and the price for the 21-inch set will decrease by 0.4 cents
    for each 19-inch set sold. How many units of each type of set should be manufactured?
\end{eg}
\begin{answer}
    Let \(x_1\) be the number of 19" TVs sold and \(x_2\) be the number of 21" TVs sold. 
    Then, our price for the 19" TVs is:
    \[
        p_1(x_1,x_2)=339-0.01x_1-0.003x_2
    \]
    And our price for the 21" TVs is:
    \[
        p_2(x_1,x_2)=399-0.004x_1-0.01x_2
    \]
    The total cost to produce TVs is:
    \[
        C(x_1,x_2)=400000+195x_1+225x_2
    \]
    Our revenue is:
    \[
        R(x_1,x_2)=x_1p_1(x_1,x_2)+x_2p_2(x_1,x_2)
    \]
    Our profit is:
    \[
        P(x_1,x_2)=R(x_1,x_2)-C(x_1,x_2)
    \]
    \includegraphics[width=0.5\linewidth]{MATH420_example2_image1.png}
    \textbf{FIX THIS WRONG EQUATION}
    \begin{note}
        When using computer algebra systems, use fractions instead of decimals
        because computers using floating point decimal systems, which can cause errors.
    \end{note}
    We are trying to maximize \(P(x_1,x_2)\) on the set:
    \[
        S=\{(x_1,x_2):x_1\geq0, x_2\leq0\}\cap \mathbb{Z}\cross\mathbb{Z}
    \] 
    \[
       \frac{\partial p}{\partial x_1}=\frac{\partial R}{\partial x_1}-\frac{\partial C}{\partial x_1} 
    \] 
    \textbf{(FINISH WRITING THIS SOLUTION, ADD JULIA PART)}

\end{answer}

What if the elasticity of 19" televisions was \(a\) instead of 0.01?

We first find \(x_1\) and \(x_2\) as functions of \(a\), and then find: \\
\[
    S(x_1, a)=\frac{a}{x_1}*\frac{dx_1}{da}|_{a=0.01, x_1 optm.}=-\frac{400}{351}
\]
and \\
\[
    S(x_2,a)=\frac{a}{x_2}*\frac{dx_2}{da}|_{a=0.01, x_2 optm.}=\frac{9695}{36153}
\]  

\begin{eg}
    We reconsider the color TV problem introduced in the previous section. There we assumed 
    that the company has the potential to produce any number of TV sets per year. Now we will 
    introduce constraints based on the available production capacity. Consideration of these 
    two new products based on the available production capacity. Consideration of these two new 
    came about because the company plans to discontinue manufacturing of some older models, thus 
    providing excess capacity at its assembly plant. This excess capacity could be sued to increase production 
    of other existing production lines, but the company feels that the new products will be more profitable 
    \textbf{FINISH WRITING}
\end{eg}

FINISH WRITING NOTES FOR SOLUTION, SIMILAR TO MATH 487

Look into math modeling competitions (SIAM, consortium)
download and learn julia. FIX FORMATTING'

\begin{eg}
    Maximize \(x+2x+3z\) over the set \(x^2+y^2+z^2=3\) \\
\end{eg}
\begin{answer}
    We have \(f(x,y,z)=x+2y+3z\) and \(g(x,y,z)=x^2+y^2+z^2\)\\
    \[
        \grad{f}=\begin{pmatrix}
            \frac{\partial f}{\partial x} \\\frac{\partial f}{\partial y} \\ \frac{\partial f}{\partial z}
        \end{pmatrix}=\begin{pmatrix}
        1 \\ 2 \\ 3
        \end{pmatrix}
        \text{ and }
    \]
\end{answer}
\textbf{FINISH WRITING ANSWER}

We now perform sensitivity analysis on our answer to this question, with \(a\) being 
the elasticity of demand: 
\[
    S(x_1, a)=\frac{dx_1}{da}*\frac{a}{x_1}
\]
\[
    S(X_2, a)= \frac{dx_2}{da}*\frac{a}{x_2} \\
\]
We first set up an equation with the Lagrange multiplier (REPHRASE): \\
\[
    \grad{P}=\lambda \grad{g} \text{ max condition}
\]
\[
    g(x_1,x_2)=x_1+x_2
\]
\[
  \text{Constraint is } g(x_1,x_2)=10000   
\]
We then solve: \\
\[
    \grad{g}=\begin{bmatrix}
    1 \\ 1
    \end{bmatrix}
\]
\[
    \grad{P}=\begin{bmatrix}
    \frac{\partial P}{\partial x_1} \\ \frac{\partial P}{\partial x_2}
    \end{bmatrix}
\]
(ADD PART WITH JULIA)\\
We get: \\
\[
    x_1=\frac{\frac{-831}{500}+\frac{13}{1000}\lambda}{\frac{49}{1000000}-\frac{1}{25}a}
\]
\[
    x_1=\frac{144-\frac{348000}{7}a-\lambda +\frac{2000}{7}a\lambda}{\frac{7}{1000}-\frac{40}{7}a}
\]\\
Sensitivity analysis: \\
\begin{align}
    S(x_1,a)&=\frac{a}{x_1}\frac{dx_1}{da}=\frac{a}{x_1}(\frac{\partial x_1}{\partial a}+\frac{\partial x_1}{\partial \lambda}\frac{d \lambda }{da}) \\
    &=\frac{a}{x_1}\big[(\frac{\frac{-831}{500}+\frac{13}{1000}\lambda }{(\frac{49}{1000000}-\frac{1}{25}a)^2})(\frac{1}{25})+\frac{13}{1000}*\frac{1}{\frac{49}{1000000}-\frac{1}{25}a}\frac{d \lambda }{da}\big]
\end{align}

We evaluate this at \(a=0.01\) and the corresponding optimal production level to get 
the elasticity at that point. 

Setting up: \\
\includegraphics[width=0.7\linewidth]{MATH420_image2.png}

Solving for maximum: \\
\includegraphics[width=0.7\linewidth]{MATH420_image3.png}

Differentiating: \\

\includegraphics[width=0.7\linewidth]{MATH420_image4.png}

\includegraphics[width=0.7\linewidth]{MATH420_image5.png}

And finally evaluating \(S(x_1,a)\): \\ 

\includegraphics[width=0.7\linewidth]{MATH420_image6.png}

We can repeat the same procedure to find \(S(x_2, a)\), but that it left as an exercise to the reader as 
I'd prefer for my notes to not be too long. 

Next, we find \(S(P, a)\) (similar workflow in Julia of finding derivatives and substituting): \\ 

\includegraphics[width=0.7\linewidth]{MATH420_image7.png}

\textbf{figure out where to add section 2, reformat and fix}

\section{Computational Methods for Optimization}
\begin{eg}
    Reconsider the pig problem of Example 1.1, but now take into account the fact that the growth rate of the pig is not constant. 
    Assume that the pig is young, so that the growth rate is increasing. When should we sell the pig 
    for maximum profit?
\end{eg}
\begin{answer}
    We start with our equations from 1.1: \\
    \[
        w(t)=200+5t
    \]
    \[
        C(t)=0.45t
    \]
    \[
        p(t)=0.65-0.01t
    \]
    \[
        R(t)=w(t)p(t)
    \]
    \[
        P(t)=R(t)-C(t)
    \]
    
    Since the rate of growth now increases with time, we rewrite the equation for weight as: \\
    \[
        w(t)=w_0e^{ct}
    \]

    We want this to be consistent with 1.1, so we solve for \(w_0\):\\
    \[
        w(t)=200=w_0e^{ct}=w_0
    \]
    
    Then, we solve for \(c\) using the fact that the pig initially grows at 5 
    pounds per day: \\
    \begin{align}
        w^{\prime} (0)&=5=w_0ce^{ct} |_{t=0}=200c \\
        \Rightarrow c&=\frac{1}{40}
    \end{align}

    We can rewrite the rate of growth in differential form: \\
    \[
        \frac{dw}{dt}=constant*w(t)
    \]

    We then set this up in Julia and plot: \\

    \includegraphics[width=0.7\linewidth]{MATH420_image8.png}

    We start our search for the maximum by finding \(\frac{dP}{dt}\): \\
    
    \includegraphics[width=0.7\linewidth]{MATH420_image9.png}

    Since this isn't linear, we can't easily solve a system of equations to find 
    the maximum. Instead, we use successive numerical approximations, called \textbf{Newton's method}: \\
    \textbf{ADD DRAWING AND EXPLANATION OF TANGENT LINE THING}

    \includegraphics[width=0.7\linewidth]{MATH420_image10.png}

    We repeat this several times, plugging the resulting back into \(g(t)\), until it converges: \\
    
    \includegraphics[width=0.7\linewidth]{MATH420_image11.png}

\end{answer}

(ask how he saves old julia things)
add section breaks, check textbook!!!!!!

install julia lab and jupyter 

Maybe make notes for each lecture in separate documents and then combine? Figure out this weekend.

\begin{definition}
    Sensitiv
\end{definition}

\end{document}