\documentclass{article}
\usepackage{graphicx} % Required for inserting images
\usepackage{tikz}
\usepackage{physics}
\usepackage{gensymb}
\usepackage{amsmath, amsfonts, mathtools, amsthm, amssymb, mathrsfs}
\usepackage{titlesec}
\usepackage{fancyhdr}
\usepackage{hyperref}
\usepackage{float}
\usepackage{booktabs}
\usepackage{enumitem}
\usepackage{subcaption}
\usepackage{multicol}
\usepackage{import}
\usepackage[usenames,dvipsnames]{xcolor}
\usepackage[a4paper, margin=1in]{geometry}
% The configuration for boxes and parts of the overall idea come from Pingbang Hu: https://github.com/sleepymalc
% Use a relative path (forward slashes or "..") to avoid backslash-escapes
\subimport{../}{header.tex}

\title{\textbf{MATH 420 Mathematical Modeling}}
\author{GoldenOrbWeaver}
\date{Spring 2026}


\begin{document}

\maketitle

\tableofcontents{\textbf{}}

\pagebreak

\pagestyle{fancy}
\fancyhead{}
\fancyhead[L]{\thetitle}
\fancyhead[R]{GoldenOrbWeaver}

\section{One Variable Optimization}
Approach to mathematical modeling:
\begin{enumerate}
    \item Ask the question
    \item Select the modeling approach
    \item Formulate the model
    \item Solve the problem
    \item Answer the question
\end{enumerate}

\begin{eg}
    A pig weighing 200 pounds gains 5 pounds per day and costs 45 cents a day 
    to keep. The market price for pigs is 65 cents per pound, but is falling
     1 center per day. When should the pig be sold?
\end{eg}

\begin{answer}
    Start at \(t=0\), with \(t\)  in days. Then we find each component:\\
    Weight of pig:   \\
    \[
        w(t)=200+5t
    \]
    Cost of pig: \[c(t)=0.45t\] \\
    Price per pound to sell pig: \[p(t)=0.65-0.01t\] \\
    Revenue from selling pig: \[r(t)=w(t)p(t)\]\\
    Profit from selling at time t: \[p(t)=r(t)-c(t)\]
\end{answer}

However, we aren't sure that our values are accurate. The amount the price
falls per day could be different, so we can instead set it to a function \(r(t)\)
then:
\[
    p(t)=0.65-rt
\]

To find the optimal time to sell, solve \(p^{\prime} (t)=0\) to get \(t_{max}\). 
We are interested in how sensitive \(t_{max}\) is on changes in \(r\), which
we call \(s(t,r)\):
\[
    s(t,v)=\lim{\frac{\frac{\Delta t}{t}}{\frac{\Delta r}{r}}}=\frac{r}{t}*\frac{dt}{dr}
\]
We begin to solve:
\[
    p^{\prime} (t)=r^{\prime} (t)-c^{\prime} (t) \\
    =w^{\prime} (t)p^{\prime} (t)=w(t)p^{\prime} (t)-0.45 \\
    =5(0.65-rt)+(200+5t)(-r)-0.45 \\
    =-10rt-200r+2.80=0
\]
Then, we solve for \(t_{max}\):
\[
    t_{max}=\frac{2.80-200r}{10r}=0.28r^{-1}-20
\] 
Next, we find \(\frac{dt_{max}}{dr}\):
\[
    \frac{dt_{max}}{dr}=-0.28r^{-2}
\] 
Therefore:
\[
    S(t,r)=\frac{-r}{t_{max}}*\frac{0.28}{r^2}=\frac{-0.28}{t_{max}r}
\]
Substituting \(t_{max}\):
\[
    S(t,r)=\frac{-0.28}{(0.28r^{-1}-20)r}=\frac{-0.28}{0.28-20r}
\] 
At our best guess of \(r=0.01\):
\[
    S(t,0.01)=\frac{-0.28}{0.28-0.2}=\frac{-0.28}{0.08}=\frac{-7}{2}
\] 
Thus, if \(r\) is wrong by \(\pm10\%\), then \(t_{max}\) will be wrong by \(S(t,r)*10\%=\frac{7}{2}*10\%=\pm35\%\) 

What if the rate of growth of the pig is also wrong? We redefine \(w(t)\):
\[
    w(t)=200+gt
\] 
Then, we find \(S(t,g)\):
\[
    S(t,g)=\frac{g}{t_{max}}*\frac{dt_{max}}{dg}|_{g=5}\approx3.0625
\] 
We can also examine the sensitivity of the outcome:
\[
    S(P,r)=\frac{r}{P}*\frac{dP}{dr}=\frac{r}{P(t_{max})}*\frac{dP(t)}{dt}*\frac{dt}{dr}|_{t=8}=0
\]
This means that the profit doesn't change that much (to the first order) when the rate of change in the price of the pig varies.

Look into math modeling competitions (SIAM, consortium)
download and learn julia. FIX FORMATTING

Maybe make notes for each lecture in separate documents and then combine? Figure out this weekend.

\begin{definition}
    Sensitiv
\end{definition}

\end{document}