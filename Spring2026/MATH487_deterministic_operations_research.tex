\documentclass{article}
\usepackage{graphicx} % Required for inserting images
\usepackage{tikz}
\usepackage{physics}
\usepackage{gensymb}
\usepackage{amsmath, amsfonts, mathtools, amsthm, amssymb, mathrsfs}
\usepackage{titlesec}
\usepackage{fancyhdr}
\usepackage{hyperref}
\usepackage{float}
\usepackage{booktabs}
\usepackage{enumitem}
\usepackage{subcaption}
\usepackage{multicol}
\usepackage{import}
\usepackage[usenames,dvipsnames]{xcolor}
\usepackage[a4paper, margin=1in]{geometry}
% The configuration for boxes and parts of the overall idea come from Pingbang Hu: https://github.com/sleepymalc
% Use a relative path (forward slashes or "..") to avoid backslash-escapes
\subimport{../}{header.tex}

\title{\textbf{MATH 487 Deterministic Operations Research}}
\author{GoldenOrbWeaver}
\date{Spring 2026}

\begin{document}

\maketitle

\tableofcontents{\textbf{}}

\pagebreak

\pagestyle{fancy}
\fancyhead{}
\fancyhead[L]{\thetitle}
\fancyhead[R]{GoldenOrbWeaver}

\section{Linear Programming}
\begin{definition}
    \textbf{Linear programming}: The optimization of a linear function subject to linear constraints.
\end{definition}

\begin{eg}
    Suppose a starving artist is trying to plan a food budget. He is health 
    conscious and wants a healthy diet that includes the following: at least
    70 g of protein per day, at least 100 g of carbohydrates per day, exactly
     15 mg of vitamin D per day, but no more than 75 g of fat per day.

     Five foods to choose from (fix formatting later): \\
    \begin{tabular}
    { |c | c | c | c | c | c | }
     \hline
     Food & Protein & Carbohydrates & Vitamin D & Fat & Cost \\
     \hline \\
        Hamburger & 10g/oz & 2g/oz & .5mg/oz & 8g/oz & \$0.20/oz \\
        Milk & 2g/oz & 3 g/oz & 4mg/oz & 2g/oz & \$0.02/oz \\
        Cereal & 3g/oz & 23g/oz & 2mg/oz & 1g/oz & \$0.10/oz \\
        Ch. N S & 2g/oz & 2g/oz & 0 mg/oz & 0.5g/oz & \$0.03/oz\\
        Eggs & 6g/egg & 4g/egg & 1mg/egg & 5g/egg & \$0.10/egg\\
     \hline

    \end{tabular}
     
    Question: How can he meet dietary goals while minimizing cost?
\end{eg}

\begin{answer}
Set up \textbf{decision variables}: \\
H, M, C, CNS, and E are oz (or number) per day\\

Constraints: \\
Protein: \(p=10H+2M+3C+2CNS+6E\geq70\) \\
Carbs: \( c=2H+3M+23C+2CNS+4E\geq100\) \\
Vitamin D: \(0.5H+4M+2C+E=15\) \\
Fat: \(f=8H+2M+1C+0.5CNS+5E\leq75\) \\
Nonnegativity: \(H,M,C,CNS,E\geq0\) \\

Subject to these requirements, we wish to minimize cost:
\[
    cost=20H+2M+10C+3CNS+10E
\]
\end{answer}

\begin{definition}
    Let \(f:\mathbb{R}^n\rightarrow \mathbb{R}\) be a function of n variables, then $f$ is called
    linear $\iff$ $f$ is of the form
    \[
        f(x_1,x_2, \dots x_n)=a_1x_1+a_2x_2+\dots+a_nx_n+b_0
    \]
    for some constraints
    \[
        a_1, a_2, \dots a_n \text{ and }b_0
    \]

\end{definition}

\begin{definition}
    A \textbf{linear equation} is an equation of the form \(f(x_1, \dots, x_n)=a\) constant. 
\end{definition}

\begin{definition}
    A \textbf{linear inequality} is an inequality of the form \(f(x_1, \dots, x_n)\leq\) 
    a constant, or \(f(x_1, \dots,x_n)\geq\) a constant.  
\end{definition}

\begin{definition}
    A \textbf{linear constraint} is either a linear equation or a linear inequality.
\end{definition}

\begin{definition}
    A \textbf{linear program} is the optimization of a linear function subject to linear constraints.
\end{definition}

\begin{eg}{The Furniture Problem}
    
    Suppose you are in charge of a furniture factory. Your plant makes tables
    and chairs out of iron, wood, and labor. \\
    \begin{tabular}
    { |c | c | c | c | c | }
     \hline
     Product & Iron (lbs) & Wood (ft) & Labor (hrs) & Profit (\$) \\
     \hline \\
        Table & 1 & 20 & 16 & 80 \\
     \hline \\
        Chair & 2 & 15 & 5 & 40 \\
     \hline
    \end{tabular} \\
    
    Suppose that your plant has access to 100 lbs of iron/day, 1000 lbs of 
    wood/day, and it has 80 employees and thus 640 labor hours/day. What should
    their production plan be?
\end{eg}

\begin{answer}
    First, we need to decide on the decision variables. These should have two properties:
    \begin{enumerate}
        \item The direction manager must have control over them
        \item Designation of optimal values solves the problem
    \end{enumerate}
    We select two, \(T\) and \(C\), the number of tables and chairs produced per day respectively. \\

    Next, we need to select our objective function. Since we wish to maximize
    profit, our objective function is:
    \[
        profit=\Pi =80T+40C
    \]

    We also need to figure out constraints:
    \[
        Iron: T+2C\leq100
        \\Wood: 20T+15C\leq1000 \\
        \\Labor: 16T+5C\leq640 
        \\Nonnegativity: T,C\geq0
    \]
    
    We have a linear program:
    \[
        max_{T,C} 80T+40C \\
        s.t. 
    \]

\end{answer}

\begin{remark}
    When there are two decision variables, we can graphically solve a linear program.
\end{remark}

ADD DRAWING later

\begin{definition}
    The \textbf{feasible region} of a linear program is the set of all points that
    satisfy all constraints.
\end{definition}

Geometrically, we wish to find the highest isoprofit that intersects the feasible
region, grazing the side of it. This will occur at a vertex (unless a constraint
line is parallel to the isoprofit line). We can check all the vertices or we can
analyze the slopes of the constraints and find the vertex of constraints with
slopes above and below the slope of the isoprofits. 

\begin{definition}
    An \textbf{integer program} is a linear program where all of the decision variables
    must have integer values.
\end{definition}

\begin{eg}{Blending Model}
    A scrap metal operator reviews an order for 24 lbs of tin, 15 lbs of copper, and 20 lbs of aluminum. 
    She can buy two types of scrap metal which she can melt down:
    \begin{tabular}
        { |c | c | c | c | c | }
        \hline
        Type & Tin & Copper & Aluminum & Cost (\$0.01/lb) \\
        \hline \\
        Metal 1 & 40\% & 50\% & 10\% & 20 \\
        \hline
        Metal 2 & 40\% & 10\% & 50\% & 10 \\
        \hline
    \end{tabular}
    Only 50 lbs of Metal 1 are available. How can she meet the order most effectively?
\end{eg}
\begin{answer}
    Decision variables: \\
    \(M_1\) = Amount of metal 1 to buy (lbs) \\
    \(M_2\) = Amount of metal 2 to buy (lbs)\\ 
    Linear program:\\
    min \(20M_1+10M_2=cost\)\\
    %\begin{align*}
    s.t. \(0.4M_1+0.4M_2\geq24\) (tin)\\
    \(0.5M_1+0.1M_2\geq15\) (copper)\\
    \(0.1M_1+0.5M_2\geq20\) (aluminum)\\
    \(M_1\leq50\) (availability)\\
    %\end{align*} FIX THIS
    Since there are only 2 decision variables, we can solve this graphically:
\end{answer}

\begin{eg}{Transportation Problems}

    Goods are located at sources and needed to be shipped to destinations. There 
    is a per unit cost to ship from any particular source to an particular 
    destination. The objective is to minimize the cost\dots

    Suppose that the Frank Perdue Chicken Co. has 2000 tons of chickens on inventory,
    500 of which are on a farm near San Francisco, 500 on a farm near Houston,
    and 1000 on a farm near Detroit. They wish to ship the chicken to four superstores 
    located in New York, Los Angeles, Kansas City, and Miami. Demand is NYC 300 tons,
    LA 900 tons, KC 600 tons, and Mia 200 tons. The shipping costs per ton are:\\
    \begin{tabular}
        { |c | c | c | c | c | }
        \hline
        From/To & NY & LA & KC & Mia \\
        \hline \\
        SF & 80 & 10 & 65 & 80 \\
        \hline
        Hou & 30 & 50 & 20 & 20 \\
        \hline
        Det & 30 & 100 & 50 & 50 \\
        \hline
    \end{tabular}

    Define \(x_{i,j}=\) the tonnage of chicken shipped from \(i\) to \(j\)\\
    Linear program: \\
    \[
      min_{x} 80x_{11}+10x_{12}+65x_{13}+80x_{14}+30x_{21}+50x_{12}+20x_{23}+20x_{24}+30x_{31}+100x_{32}+50x_{33}+50x_{34}     
    \]
    s.t. \(x_{11}+x_{12}+x_{13}+x_{14}\leq500\)\\
    \(x_{21}+x_{22}+x_{23}+x_{24}\leq500\)\\
    \(x_{31}+x_{32}+x_{33}+x_{34}\leq1000\)\\
    \(x_{11}+x_{21}+x_{31}\geq300\)\\
    \(x_{12+x_{22}+x_{32}\geq900}\)\\
    \(x_{13}+x_{23}+x_{33}\geq600\)\\
    \(x_{14}+x_{24}+x_{34}\geq200\)\\  
    \(x_{i,j}\geq0\) 
    \begin{remark}
        Note the specific special structure of the constraint matrix. This allows
        for specialized algorithms to solve transportation problems.
    \end{remark}
    \begin{remark}
        In this particular problem, the sum of the supplies at sources equals the 
        sum of the demands at destinations. This implies that for any feasible solution, 
        all the constraints hold with equality. In general transportation problems, 
        the total supply at sources is greater than or equal to total demand at sinks.
    \end{remark}
    \begin{remark}
        In general, the transportation problem has the form: \\
        \(max_{x}\)  \(\sum_{i=1}^{I}\sum_{j=1}^{J}C_{ij}X_{ij}\) \\
        s.t. \(\sum_{i=1}^{I}X_{ij}\leq S_i \) for \(i=1,\dots,I\) \\
        \(\sum_{j=1}^{J}X_{ij}\geq D_j \) for \(j=1,\dots,J\) \\
        \(X_{ij}\geq0\) for all \(i,j\) \\
        where \(C_{ij}=\) cost per unit shipped from source \(i\) to destination \(j\) \\ 
        \(I=\) number of sources \\
        \(J=\) number of destinations \\
        \(S_i=\) on hand at source i \\
        \(d_j=\) demand at destination j    
    \end{remark}
\end{eg}

\end{document}