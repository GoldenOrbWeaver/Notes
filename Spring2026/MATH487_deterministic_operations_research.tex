\documentclass{article}
\usepackage{graphicx} % Required for inserting images
\usepackage{tikz}
\usepackage{physics}
\usepackage{gensymb}
\usepackage{amsmath, amsfonts, mathtools, amsthm, amssymb, mathrsfs}
\usepackage{titlesec}
\usepackage{fancyhdr}
\usepackage{hyperref}
\usepackage{float}
\usepackage{booktabs}
\usepackage{enumitem}
\usepackage{subcaption}
\usepackage{multicol}
\usepackage{import}
\usepackage[usenames,dvipsnames]{xcolor}
\usepackage[a4paper, margin=1in]{geometry}
% The configuration for boxes and parts of the overall idea come from Pingbang Hu: https://github.com/sleepymalc
% Use a relative path (forward slashes or "..") to avoid backslash-escapes
\subimport{../}{header.tex}

\title{\textbf{MATH 487 Deterministic Operations Research}}
\author{GoldenOrbWeaver}
\date{Spring 2026}

\begin{document}

\maketitle

\tableofcontents{\textbf{}}

\pagebreak

\pagestyle{fancy}
\fancyhead{}
\fancyhead[L]{\thetitle}
\fancyhead[R]{GoldenOrbWeaver}

\section{Linear Programming}
\begin{definition}
    Linear programming: The optimization of a linear function subject to linear constraints.
\end{definition}

\begin{eg}
    Suppose a starving artist is trying to plan a food budget. He is health 
    conscious and wants a healthy diet that includes the following: at least
    70 g of protein per day, at least 100 g of carbohydrates per day, exactly
     15 mg of vitamin D per day, but no more than 75 g of fat per day.

     Five foods to choose from (fix formatting later): \\
     Hamburger: 10g protein/oz, 2g carb/oz, .5mg vit D/oz, 8g fat/oz, \$0.20/oz \\
     Milk: 2g protein/oz, 3 g carb/oz, 4mg vit D/oz, 2g fat/oz, \$0.02/oz \\
     Cereal: 3g protein/oz, 23g carb/oz, 2mg vit D/oz, 1g fat/oz, \$0.10/oz \\
     Ch. N S: 2g protein/oz, 2g carb/oz, 0 vit D/oz, 0.5g fat/oz, \$0.03/oz\\
     Eggs: 6g protein/egg, 4g carb/egg, 1mg vit D/egg, 5g fat/egg, \$0.10/egg\\

     Question: How can he meet dietary goals while minimizing cost?
\end{eg}

Set up variables: \\
H, M, C, CNS, and E are oz (or number) per day, called \textbf{decision variables}\\

Constraints: \\
Protein: \(p=10H+2M+3C+2CNS+6E\geq70\) \\
Carbs: \( c=2H+3M+23C+2CNS+4E\geq100\) \\
Vitamin D: \(0.5H+4M+2C+E=15\) \\
Fat: \(f=8H+2M+1C+0.5CNS+5E\leq75\) \\
Nonnegativity: \(H,M,C,CNS,E\geq0\) \\

Subject to these requirements, we wish to minimize cost:
\[
    cost=20H+2M+10C+3CNS+10E
\]
\begin{definition}
    Let \(f:R^n\rightarrow R\) be a function of n variables, then f is called
    linear $\iff$ f is of the form
    \[
        f(x_1,x_2, \dots x_n)=a_1x_1+a_2x_2+\dots+a_nx_n+b_0
    \]
    for some constraints
    \[
        a_1, a_2, \dots a_n \text{and}b_0
    \]

\end{definition}

\begin{definition}
    A linear equation is an equation of the form \(f(x_1, \dots, x_n)=a\) constant. 
\end{definition}

\begin{definition}
    A linear inequality is an inequality of the form \(f(x_1, \dots, x_n)\leq\) 
    a constant, or \(f(x_1, \dots,x_n)\geq\) a constant.  
\end{definition}

\begin{definition}
    A linear constraint is either a linear equation or a linear inequality.
\end{definition}

\begin{definition}
    A linear program is the optimization of a linear function subject to linear constraints.
\end{definition}


\end{document}