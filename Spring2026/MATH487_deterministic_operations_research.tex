\documentclass{article}
\usepackage{graphicx} % Required for inserting images
\usepackage{tikz}
\usepackage{physics}
\usepackage{gensymb}
\usepackage{amsmath, amsfonts, mathtools, amsthm, amssymb, mathrsfs}
\usepackage{titlesec}
\usepackage{fancyhdr}
\usepackage{hyperref}
\usepackage{float}
\usepackage{booktabs}
\usepackage{enumitem}
\usepackage{subcaption}
\usepackage{multicol}
\usepackage{import}
\usepackage[usenames,dvipsnames]{xcolor}
\usepackage[a4paper, margin=1in]{geometry}
% The configuration for boxes and parts of the overall idea come from Pingbang Hu: https://github.com/sleepymalc
% Use a relative path (forward slashes or "..") to avoid backslash-escapes
\subimport{../}{header.tex}

\title{\textbf{MATH 487 Deterministic Operations Research}}
\author{GoldenOrbWeaver}
\date{Spring 2026}

\begin{document}

\maketitle

\tableofcontents{\textbf{}}

\pagebreak

\pagestyle{fancy}
\fancyhead{}
\fancyhead[L]{\thetitle}
\fancyhead[R]{GoldenOrbWeaver}

\section{Linear Programming}
\begin{definition}
    \textbf{Linear programming}: The optimization of a linear function subject to linear constraints.
\end{definition}

\begin{eg}
    Suppose a starving artist is trying to plan a food budget. He is health 
    conscious and wants a healthy diet that includes the following: at least
    70 g of protein per day, at least 100 g of carbohydrates per day, exactly
     15 mg of vitamin D per day, but no more than 75 g of fat per day.

     Five foods to choose from (fix formatting later): \\
    \begin{tabular}
    { |c | c | c | c | c | c | }
     \hline
     Food & Protein & Carbohydrates & Vitamin D & Fat & Cost \\
     \hline \\
        Hamburger & 10g/oz & 2g/oz & .5mg/oz & 8g/oz & \$0.20/oz \\
        Milk & 2g/oz & 3 g/oz & 4mg/oz & 2g/oz & \$0.02/oz \\
        Cereal & 3g/oz & 23g/oz & 2mg/oz & 1g/oz & \$0.10/oz \\
        Ch. N S & 2g/oz & 2g/oz & 0 mg/oz & 0.5g/oz & \$0.03/oz\\
        Eggs & 6g/egg & 4g/egg & 1mg/egg & 5g/egg & \$0.10/egg\\
     \hline

    \end{tabular}
     
    Question: How can he meet dietary goals while minimizing cost?
\end{eg}

\begin{answer}
Set up \textbf{decision variables}: \\
H, M, C, CNS, and E are oz (or number) per day\\

Constraints: \\
Protein: \(p=10H+2M+3C+2CNS+6E\geq70\) \\
Carbs: \( c=2H+3M+23C+2CNS+4E\geq100\) \\
Vitamin D: \(0.5H+4M+2C+E=15\) \\
Fat: \(f=8H+2M+1C+0.5CNS+5E\leq75\) \\
Nonnegativity: \(H,M,C,CNS,E\geq0\) \\

Subject to these requirements, we wish to minimize cost:
\[
    cost=20H+2M+10C+3CNS+10E
\]
\end{answer}

\begin{definition}
    Let \(f:\mathbb{R}^n\rightarrow \mathbb{R}\) be a function of n variables, then $f$ is called
    linear $\iff$ $f$ is of the form
    \[
        f(x_1,x_2, \dots x_n)=a_1x_1+a_2x_2+\dots+a_nx_n+b_0
    \]
    for some constraints
    \[
        a_1, a_2, \dots a_n \text{ and }b_0
    \]

\end{definition}

\begin{definition}
    A \textbf{linear equation} is an equation of the form \(f(x_1, \dots, x_n)=a\) constant. 
\end{definition}

\begin{definition}
    A \textbf{linear inequality} is an inequality of the form \(f(x_1, \dots, x_n)\leq\) 
    a constant, or \(f(x_1, \dots,x_n)\geq\) a constant.  
\end{definition}

\begin{definition}
    A \textbf{linear constraint} is either a linear equation or a linear inequality.
\end{definition}

\begin{definition}
    A \textbf{linear program} is the optimization of a linear function subject to linear constraints.
\end{definition}

\begin{eg}{The Furniture Problem}
    
    Suppose you are in charge of a furniture factory. Your plant makes tables
    and chairs out of iron, wood, and labor. \\
    \begin{tabular}
    { |c | c | c | c | c | }
     \hline
     Product & Iron (lbs) & Wood (ft) & Labor (hrs) & Profit (\$) \\
     \hline \\
        Table & 1 & 20 & 16 & 80 \\
     \hline \\
        Chair & 2 & 15 & 5 & 40 \\
     \hline
    \end{tabular} \\
    
    Suppose that your plant has access to 100 lbs of iron/day, 1000 lbs of 
    wood/day, and it has 80 employees and thus 640 labor hours/day. What should
    their production plan be?
\end{eg}

\begin{answer}
    First, we need to decide on the decision variables. These should have two properties:
    \begin{enumerate}
        \item The direction manager must have control over them
        \item Designation of optimal values solves the problem
    \end{enumerate}
    We select two, \(T\) and \(C\), the number of tables and chairs produced per day respectively. \\

    Next, we need to select our objective function. Since we wish to maximize
    profit, our objective function is:
    \[
        profit=\Pi =80T+40C
    \]

    We also need to figure out constraints:
    \[
        Iron: T+2C\leq100
        \\Wood: 20T+15C\leq1000 \\
        \\Labor: 16T+5C\leq640 
        \\Nonnegativity: T,C\geq0
    \]
    
    We have a linear program:
    \[
        max_{T,C} 80T+40C \\
        s.t. 
    \]

\end{answer}

\begin{remark}
    When there are two decision variables, we can graphically solve a linear program.
\end{remark}

ADD DRAWING later

\begin{definition}
    The \textbf{feasible region} of a linear program is the set of all points that
    satisfy all constraints.
\end{definition}

Geometrically, we wish to find the highest isoprofit that intersects the feasible
region, grazing the side of it. This will occur at a vertex (unless a constraint
line is parallel to the isoprofit line). We can check all the vertices or we can
analyze the slopes of the constraints and find the vertex of constraints with
slopes above and below the slope of the isoprofits. 

\begin{definition}
    An \textbf{integer program} is a linear program where all of the decision variables
    must have integer values.
\end{definition}

\begin{eg}{Blending Model}
    A scrap metal operator reviews an order for 24 lbs of tin, 15 lbs of copper, and 20 lbs of aluminum. 
    She can buy two types of scrap metal which she can melt down:
    \begin{tabular}
        { |c | c | c | c | c | }
        \hline
        Type & Tin & Copper & Aluminum & Cost (\$0.01/lb) \\
        \hline \\
        Metal 1 & 40\% & 50\% & 10\% & 20 \\
        \hline
        Metal 2 & 40\% & 10\% & 50\% & 10 \\
        \hline
    \end{tabular}
    Only 50 lbs of Metal 1 are available. How can she meet the order most effectively?
\end{eg}
\begin{answer}
    Decision variables: \\
    \(M_1\) = Amount of metal 1 to buy (lbs) \\
    \(M_2\) = Amount of metal 2 to buy (lbs)\\ 
    Linear program:\\
    min \(20M_1+10M_2=cost\)\\
    %\begin{align*}
    s.t. \(0.4M_1+0.4M_2\geq24\) (tin)\\
    \(0.5M_1+0.1M_2\geq15\) (copper)\\
    \(0.1M_1+0.5M_2\geq20\) (aluminum)\\
    \(M_1\leq50\) (availability)\\
    %\end{align*} FIX THIS
    Since there are only 2 decision variables, we can solve this graphically:
\end{answer}

\begin{eg}{Transportation Problems}

    Goods are located at sources and needed to be shipped to destinations. There 
    is a per unit cost to ship from any particular source to an particular 
    destination. The objective is to minimize the cost\dots

    Suppose that the Frank Perdue Chicken Co. has 2000 tons of chickens on inventory,
    500 of which are on a farm near San Francisco, 500 on a farm near Houston,
    and 1000 on a farm near Detroit. They wish to ship the chicken to four superstores 
    located in New York, Los Angeles, Kansas City, and Miami. Demand is NYC 300 tons,
    LA 900 tons, KC 600 tons, and Mia 200 tons. The shipping costs per ton are:\\
    \begin{tabular}
        { |c | c | c | c | c | }
        \hline
        From/To & NY & LA & KC & Mia \\
        \hline \\
        SF & 80 & 10 & 65 & 80 \\
        \hline
        Hou & 30 & 50 & 20 & 20 \\
        \hline
        Det & 30 & 100 & 50 & 50 \\
        \hline
    \end{tabular}

    Define \(x_{i,j}=\) the tonnage of chicken shipped from \(i\) to \(j\)\\
    Linear program: \\
    \[
      min_{x} 80x_{11}+10x_{12}+65x_{13}+80x_{14}+30x_{21}+50x_{12}+20x_{23}+20x_{24}+30x_{31}+100x_{32}+50x_{33}+50x_{34}     
    \]
    s.t. \(x_{11}+x_{12}+x_{13}+x_{14}\leq500\)\\
    \(x_{21}+x_{22}+x_{23}+x_{24}\leq500\)\\
    \(x_{31}+x_{32}+x_{33}+x_{34}\leq1000\)\\
    \(x_{11}+x_{21}+x_{31}\geq300\)\\
    \(x_{12+x_{22}+x_{32}\geq900}\)\\
    \(x_{13}+x_{23}+x_{33}\geq600\)\\
    \(x_{14}+x_{24}+x_{34}\geq200\)\\  
    \(x_{i,j}\geq0\) 
    \begin{remark}
        Note the specific special structure of the constraint matrix. This allows
        for specialized algorithms to solve transportation problems.
    \end{remark}
    \begin{remark}
        In this particular problem, the sum of the supplies at sources equals the 
        sum of the demands at destinations. This implies that for any feasible solution, 
        all the constraints hold with equality. In general transportation problems, 
        the total supply at sources is greater than or equal to total demand at sinks.
    \end{remark}
    \begin{remark}
        In general, the transportation problem has the form: \\
        \(max_{x}\)  \(\sum_{i=1}^{I}\sum_{j=1}^{J}C_{ij}X_{ij}\) \\
        s.t. \(\sum_{i=1}^{I}X_{ij}\leq S_i \) for \(i=1,\dots,I\) \\
        \(\sum_{j=1}^{J}X_{ij}\geq D_j \) for \(j=1,\dots,J\) \\
        \(X_{ij}\geq0\) for all \(i,j\) \\
        where \(C_{ij}=\) cost per unit shipped from source \(i\) to destination \(j\) \\ 
        \(I=\) number of sources \\
        \(J=\) number of destinations \\
        \(S_i=\) on hand at source i \\
        \(d_j=\) demand at destination j    
    \end{remark}
\end{eg}

\begin{eg}
    Daisy Drugs manufactures two drugs, Drug 1 and Drug 2. The drugs are produced
    by blending together two chemicals: Chemical ! and Chemical 2. By weight,
    Drug 1 must contain at least 65\% Chemical 1, and Drug 2 must contain at least
    55 \% Chemical 1. Both drugs, by weight, are completely composed of Chemical 1
    and Chemical 2. Drug 1 sells for \$6/oz and Drug 2 sells for \$4/oz. Chemicals  1
    and 2 are produced by one of two production processes: \\
    \begin{tabular}
        { |c | c | c | c |c | } 
        \hline
        Process & Raw Mat./hr & Labor/hr & Chemical 1 & Chemical 2 \\
        \hline \\
        I & 3 & 2 & 3 & 3 \\
        \hline 
        II & 2 & 3 & 3 & 1 \\
        \hline
    \end{tabular} \\

    A total of 120 hours of skilled labor, 100 oz of raw material are available. 
    Formulate a LP which can be used to maximize Daisy's sale revenue. 
    \begin{remark}
        There is a more complicated problem than the furniture problem because 
        we now have to model a multi-step production process. We have 1 (raw materials 
        and labor) -> 2 (chemicals) -> 3 (drugs).
    \end{remark}

    Decision variables: \\
    \(D_1=\) amount of Drug 1 produced (oz) \\
    \(D_2=\) amount of Drug 2 produced (oz) \\
    \(C_1=\) amount of Chemical 1 produced (oz) \\
    \(C_2=\) amount of Chemical 2 produced (oz) \\
    \(P_1=\) number of hours used with Process I \\
    \(P_2=\) number of hours used with Process II \\ 
    \(C_1D_1=\) amount of Chemical 1 used in Drug 1 (oz) \\
    \(C_2D_1=\) amount of Chemical 2 used in Drug 1 (oz) \\
    \(C_1D_2=\) amount of Chemical 1 used in Drug 2 (oz) \\
    \(C_2D_2=\) amount of Chemical 2 used in Drug 2 (oz) \\  

    Linear program: \\
    max \(6D_1+4D_2\) \\
    s.t. \(\frac{C_1D_1}{C_1D_1+C_2D_1}\geq0.65\) \\
    \(=>C_1D_1\geq0.65(C_1D_1+C_2D_1)\) \\
    \(=>0.35C_1D_1-0.65C_2D_1\geq0\) \\
    \(0.45C_1D_2-0.55C_2D_2\geq0\) (Since we need constraints to be linear) \\
    \(C_1-C_1D_1-C_1D_2\geq0\) \\
    \(C_2-C_2D_1-C_2D_2\geq0\) \\
    \(D_1=C_1D_1+C_2D_2\) \\
    \(D_2=C_1D_2+C_2D_2\) \\

    The constraints defining \(C_1\), \(C_1\), \(D_1\), and \(D_2\) are sometimes called 
    linking constraints. We still need constraints linking 1 to 2:

    \(C_1-3P_1-3P_2=0\) \\
    \(C_2-3P_1-P_2=0\)  \\

    We also have resource constraints: \\
    \(3P_1+2P_2\leq100\) \\
    \(2P_1+3P_2\leq120\)  \\

    And of course nonnegativity: \\
    \(D_1, D_2, C_1, C_2, C_1D_1, C_1D_2, C_2D_1, C_2D_2, P_1, P_2 \geq 0\) 

    \begin{definition}
        \textbf{The Simplex Method} is a way of solving an LP with more than 2 decision variables\dots
        
        Preliminaries: \\
        \begin{enumerate}
            \item 3 Classes of LPs 
            \begin{enumerate}
                \item LPs with finite optima
                \item Infeasible LPs, where \(F.R.=\emptyset\) 
                \item Unbounded LPs, where either:
                \begin{enumerate}
                    \item min problems where we can send \(z^*\rightarrow-\infty\) 
                    \item max problems where we can send \(z^*\rightarrow\infty\) 
                \end{enumerate}
            \end{enumerate}
            We desire that an algorithm tells us which of these classes our LP is in 
            and if an optimum exists, what it is.
        \end{enumerate}
    \end{definition}
\end{eg}

Types of linear programs: \\
\begin{enumerate}
    \item LPs in general form
    \begin{enumerate}
        \item max or min problem
        \item constraints are \(\leq\), \(\geq\), or \(=\) (never < or >) 
        \item RHS of constraints can be negative, positive or O 
        \item The variables can be nonnegative, nonpositive, or unrestricted
    \end{enumerate}
    \item LPs in standard form. Similar to LPs in general form except:
    \begin{enumerate}
        \item Constraints are all \(=\)
        \item RHS of constraints are all positive or 0
        \item Variables are all nonnegative 
    \end{enumerate}
\end{enumerate}

\begin{proposition}
    Any LP in general form can be rewritten as an equivalent LP in standard form.
\end{proposition}
\begin{proof}
    We demonstrate a procedure on an example, which can be used on any example. The 
    steps can be used in any order. 

    Put the LP below into an equivalent LP in standard form:\\
    \(max \text{ } 6x_1+x_2-x_3-x_4=z\)\\
    s.t. \(x_1-x_2\leq5\) \\
    \(x_1+2x_2+x_3-x_4\geq 2\) \\
    \(2x_2+x_3=-4\) \\
    \(x_1, x_4 \geq 0\) \\
    \(x_2\leq0\) \\
    \(x_3 \text { unrestricted}\)
    
    Step 1: Change "max" problem to "min" problem, if necessary, by multiplying the 
    objective function by -1: \\
    \(min \text{} -6x_1-x_2+x_3+x_4=-z\)
    s.t. \(x_1-x_2\leq5\) \\
    \(x_1+2x_2+x_3-x_4\geq 2\) \\
    \(2x_2+x_3=-4\) \\
    \(x_1, x_4 \geq 0\) \\
    \(x_2\leq0\) \\
    \(x_3 \text { unrestricted}\) 

    Step 2: Multiply any constraints with negative RHS by -1. (This changes 
    \(\leq\) constraints to \(\geq\) and vice versa): \\
    \(min \text{} -6x_1-x_2+x_3+x_4=-z\)
    s.t. \(x_1-x_2\leq5\) \\
    \(x_1+2x_2+2x_3-x_4\geq2\) \\
    \(-2x_2-x_3=4\) \\
    \(x_1, x_4 \geq 0\) \\
    \(x_2\leq 0\) \\
    \(x_3 \text{ unrestricted}\) \\
    
    Step 3: Add slack variables so that all constraints are \(=\): \\
    \(min \text{} -6x_1-x_2+x_3+x_4=-z\)
    s.t. \(x_1-x_2+S_1=5\) \\
    \(x_1+2x_2+2x_3-x_4-S_2=2\) \\
    \(-2x_2-x_3=4\) \\
    \(x_1, x_4\geq 0\) \\
    \(x_2\leq 0\) \\
    \(x_3 \text{ unrestricted}\) \\
    \(S_1, S_2\geq 0\) \\
    
    Step 4: Replace any nonpositive variable \(x_i\) with a new variable 
    \(x_i^*\), where \(x_i^{\prime} =-x_i\): \\
    \(min \text{} -6x_1+x_2^{\prime} +x_3+x_4=-z\)
    s.t. \(x_1+x_2^{\prime} +S_1=5\) \\
    \(x_1-2x_2^{\prime} +2x_3-x_4-S_2=2\) \\
    \(2x_2^{\prime} -x_3=4\) \\
    \(x_1, x_2^{\prime}, x_4, S_1, S_2 \geq 0\) \\
    \(x_3 \text{ unrestricted}\) \\

    Step 5: Replace any unrestricted variables \(x_i\) with the difference of two 
    nonnegative variables \(x_i^{\prime} -x_i^{\prime\prime} \): \\
    \(min \text{} -6x_1+x_2^{\prime} +x_3^{\prime} -x_3^{\prime\prime} +x_4=-z\)
    s.t. \(x_1+x_2^{\prime} +S_1=5\) \\
    \(x_1-2x_2^{\prime} +2x_3^{\prime} -2x_3^{\prime\prime} -x_4-S_2=2\) \\
    \(2x_2^{\prime} -x_3^{\prime} +x_3^{\prime\prime} =4\) \\
    \(x_1, x_2^{\prime}, x_3^{\prime} , x_3^{\prime\prime} x_4, S_1, S_2 \geq 0\) \\

    This is now in standard form

\end{proof}

\begin{note}
    Suppose we solved the linear program in standard form that we just found 
    using the simplex method and get an optimal answer of: \\
    \(x_1=0\), \(x_2^{\prime} =5\), 
    \(x_3^{\prime} =6\), \(x_3^{\prime\prime} =0\), \(x_4=0\), \(S_1=0\), \(S_2=0\), 
    \(-z=11\)\\
    Then we can translate back and find that the solution of the linear 
    program in general form is: \\
    \(x_1=0\), \(x_2=-5\), \(x_3=6\), \(x_4=0\), \(z=-11\)     
\end{note}

\begin{definition}
    An \(m\cross n\) linear program is in \textbf{canonical form} if it is 
    in standard form and there is a distinguished set of \(m\) variables called 
    \textbf{basic variables} for which: \\
    \begin{enumerate}
        \item Each basic variable has coefficient 1 in one constraint and 0 
        in the others. 
        \item Each basic variable has coefficient 0 in the objective function. 
        \item Each constraint has exactly 1 basic variable with coefficient 1.
    \end{enumerate}
\end{definition}

\begin{eg}
    Recall the furniture problem: \\
    \(\text{max }80T+40C=z\) \\
    s.t. \(T+2C\leq100\) \\
    \(20T+15C\leq1000\) \\
    \(16T+5C\leq640\) \\
    \(T,C\geq0\)\\
    
    We start by converting this to a linear program in standard form: \\
    \(\text{min }-80T-40C=-z\) \\
    s.t \(T+2C+S_1=100\) \\
    \(20T+15C+S_2=1000\) \\
    \(16T+5C+S_3=640\) \\
    \(T, C, S_1, S_2\geq0\) \\
    
    This happens to also be in canonical form with the set of basic variables 
    or basis=\(\left\{S_1, S_2, S_3\right\}\)     
\end{eg}

\begin{remark}
    We were lucky that our standard form turned out to be a canonical form. In most 
    examples this won't be the case.
\end{remark} 

\begin{remark}
    If we have a canonical form LP, it is easy to see what is called an associated \textbf{basic feasible solution}. 
    In this case, \(T=C=0\) (nonbasic variables are 0), \(S_1=100\), \(S_2=1000\), and \(S_3=640\) (basic variables are RHS).    
\end{remark}

We now have a b.f.s. for the LP, but this isn't optimal. We can raise C while reducing 
\(S_1\), \(S_2\), \(S_3\) to maintain constraints, but we can't do this indefinitely, only 
up to \(\frac{100}{2}\) for the iron constraint, \(\frac{1000}{15}\) for the wood constraint, 
and \(\frac{640}{5}\) for the labor constraint. Therefore, we should make \(C\) basic and \(S_1\) nonbasic.

\begin{theorem}{Minimum-Ratio Test}\\
    In general, if we've decided that \(x_{j^*}\) should enter the basis, then the current basic 
    variable of the \(i^*\)th row should leave the basis where \(i^*\) minimizes \(\frac{b_i}{a_{ij^*}}\) over 
    all \(i\) s.t. \(a_{ij^*}>0\) .    
\end{theorem}

To make \(C\) basic and \(S_1\) nonbasic while retaining canonical form:
\begin{enumerate}
    \item Multiply equation (1) by \(1/2\), obtaining (1').
    \item Subtract \(15*(1^{\prime})\)  from (2), obtaining (2')\\
    Subtract \(15*(1^{\prime})\) from (3), obtaining (3') \\
    Subtract \(-40*(1^{\prime})\) from (obj), obtaining (obj') 
\end{enumerate}
Doing the first step: \\
\[
    \frac{1}{2}T+C+\frac{1}{2}S_1=50
\]  
Doing the second step: \\
\begin{align}
    20T+15C+S_2&=1000 \\
    -15(\frac{1}{2}T+C+\frac{1}{2}S_1&=50 \\
    \frac{25}{2}T-\frac{15}{2}S_1+S_2&=250
\end{align}
\begin{align}
    16T+5C+S_3&=640 \\
    -5(\frac{1}{2}T+C+\frac{1}{2}S_1&=50) \\
    \frac{27}{2}T-\frac{5}{2}S_1+S_3&=390
\end{align}
\begin{align}
    -80T-40C&=-Z \\
    +40(\frac{1}{2}T+C+\frac{1}{2}S_1&=50) \\
    -60T+20S_1&=-Z+2000
\end{align}
Our new LP is: \\
\(\text{max }-60T+20S_1=-Z+2000\) \\
s.t. \(\frac{1}{2}T+C+\frac{1}{2}S_1=50\)\\
\(\frac{25}{2}T-\frac{15}{2}S_1+S_2=250\) \\
\(\frac{27}{2}T-\frac{5}{2}S_1+S_3=390\) \\
\(T, C, S_1, S_2, S_3\geq0\)\\


We have a new canonical form with basis \(C, S_2, S_3\). The associated b.f.s 
is \((T,C,S_1,S_2,S_3)=(0, 50, 0, 250, 390)\).

Note that we can read the associated \(Z\) value from the canonical form (here \(Z=2000\).) 

\begin{remark}
    In general, if we wish to enter \(x_{j*}\) into the basis and remove the basic variable 
    from the \(i^*\)th row from the basis, then the canonical form tableau changes as follows: \\
    \begin{note}
        No prime means before pivot and prime means after pivot.
    \end{note}
    \begin{enumerate}
        \item \(a_{i^*j^*}^{\prime}=1\) 
        \item \(a_{i^*j}^{\prime} =\frac{a_{i^*j}}{a_{i^*j^*}}\) for \(j=j^*\)
        \item \(a_{ij^*}^{\prime} =0\) for \(i\neq i^*\)
        \item \(a_{ij}^{\prime} =a_{ij}-\frac{a_{ij^*}*a_{i^*j}}{a_{i^*j^*}}\) for \(i\neq i^*\) and \(j\not j^*\)
        \item \(b_{i^*}^{\prime} =\frac{b_{i^*}}{a_{i^*j^*}}\)
        \item \(b_i^{\prime} =b_i-\frac{b_{i^*a_{ij^*}}}{a_{i^*j^*}}\) for \(i\neq i^*\)
        \item \(C_{j^*}^{\prime} =0\) 
        \item \(C_j^{\prime} =C_j-\frac{C_{j^*}a_{i^*j}}{a_{i^*j^*}}\) for \(j\neq j^*\)
        \item \((\pm Z)+Z_0^{\prime}=(\pm Z)+Z_0-\frac{b_{i^*}C_{j^*}}{a_{i^*j^*}}\)    
    \end{enumerate} 
    The operation which generates a new canonical form from an old canonical form is called a \textbf{pivot} on \(a_{i^*j^*}\) 
\end{remark}

An easy way to do this is to use the \textbf{box method}. First, write the original LP in 
\textbf{tableau form}: \\
\begin{tabular}{c c c c c | c}
    T & C & $S_1$ & $S_2$ & $S_3$ & RHS \\
    1 & 2 & 1 & 0 & 0 & 100 \\
\end{tabular}

\textbf{FINISH ADDING PHYSICAL NOTES}

\begin{definition}{Simplex Algorithm}

    Given an LP problem in general form:
    \begin{enumerate}
        \item Turn into an equivalent problem in standard form
        \item Obtain an initial canonical form, or else determine that the LP is infeasible. 
        \item \begin{enumerate}
            \item Consider the coefficients in the objective row of the current canonical form. If they are all nonnegative, 
            the current associate B.F.S. is optimal. If not, select a column (\(j^*\) ) in which 
            the coefficient is negative, to pivot \(x_{j^*}\) into the basis.
            \item If \(a_{ij^*}\leq0\) for \(i=i,\dots,m\), the LP is unbounded. Otherwise, 
            perform the M.R.T. to determine the coefficient \(a_{i^*j^*}\) on which to pivot next.
            \item Pivot on \(a_{i^*j^*}\), obtaining a new canonical form. Go to (a).    
        \end{enumerate}
    \end{enumerate}
\end{definition}

Suppose we are given an LP in standard form: \\
\begin{align}
    \text{min }c_1x_1+\dots+c_n x_n &=z \\
    \text{s.t. }a_{11}x_1+\dots+a_{1n}x_n&=b_1 \\
    \vdots \\
    a_{m1}x_1+\dots+a_{mn}x_n&=b_m \\
    x_{1},\dots, x_ n & \geq0
\end{align}

Let us introduce \(m\) artificial variables \(w_1,\dots,w_m\) and consider a second 
objective function as follows: \\
\begin{align}
    \text{min }c_1x_1+\dots+c_n x_n &=z \\
    \text{min }w_1+\dots+w_m &=w \\
    \text{s.t. }a_{11}x_1+\dots+a_{1n}x_n+w_1&=b_1 \\
    \vdots \\
    a_{m1}x_1+\dots+a_{mn}x_n+w_m&=b_m \\
    x_{1},\dots,x_ n, w_1, \dots, w_m & \geq0
\end{align}

We note two things: \\
\begin{enumerate}
    \item The original LP is feasible iff the optimal \(w^*\) is 0.
    \item The LP (where \(w\) is the objective) is almost in canonical form. The only 
    problem is that we have \(w_i\)s in the objective function. 
\end{enumerate}

To fix (2), we use the substitutions:
\begin{align}
    w_1=b_1-a_{11}x_1-a_{12}x_2-\dots-a_{1n}x_n \\
    \vdots \\
    w_m=b_m-a_{1m}x_1-a_{m2}x_2-\dots-a_{mn}x_n \\
\end{align}

So the second objective function is now: \\
\begin{align}
    \text{min }w&=w_1+\dots+w_m \\
    &= \sum_{i=1}^m b_1-(\sum_{i=1}^m a_{11})x_1-\dots -(\sum_{i=1}^m a_{1n})x_n
\end{align}

We rewrite this: \\
\[
    \text{min }-(\sum_{i=1}^m a_{11})x_1-\dots -(\sum_{i=1}^m a_{1n})x_n=w-\sum_{i=1}^m b_1
\]

So, part 2 of the simplex method is: Given the LP in standard for, set up the following 
LP: \\
\begin{align}
    \text{min }c_1x_1+\dots+c_n x_n &=z \\
    \text{min }-(\sum_{i=1}^m a_{11})x_1-\dots -(\sum_{i=1}^m a_{1n})x_n &=w-\sum_{i=1}^m b_1 \\
    \text{s.t. }a_{11}x_1+\dots+a_{1n}x_n+w_1&=b_1 \\
    \vdots \\
    a_{m1}x_1+\dots+a_{mn}x_n+w_m&=b_m \\
    x_{1},\dots,x_ n, w_1, \dots, w_m & \geq0
\end{align}

This is in canonical form with basis \(w_1,\dots,w_m\)

We solve this to find the optimal \(w^*\). We have two scenarios: \\
\begin{enumerate}
    \item If \(w^*>0\), the LP is infeasible \\
    \item If \(w^*=0\), the LP is feasible and you have a canonical form 
\end{enumerate} 

\begin{eg}
    \begin{align}
        \text{min }3x_1+x_2-x_3-x_4 &=z \\
        \text{s.t. }x_1-x_2+x_3+2x_4 &=3 \\
        2x_1+2x_2-x_3-x_4 &=1 \\
        -x_1+x_2+3x_3+3x_4 &=6 \\
        x_1, x_2, x_3, x_4 &\geq 0
    \end{align}
\end{eg}
\begin{answer}
    First, we add \(m=3\) artificial variables \(w_1, w_2, w_3\): \\
    \begin{align}
        \text{min }3x_1+x_2-x_3-x_4 &=z \\
        \text{min }-2x_1-2x_2-3x_3+x_4 &=w-10 \\
        \text{s.t. }x_1-x_2+x_3+2x_4 +w_1 &=3 \\
        2x_1+2x_2-x_3-x_4 +w_2&=1 \\
        -x_1+x_2+3x_3+3x_4 +w_3&=6 \\
        x_1, x_2, x_3, x_4, w_1, w_2, w_3 &\geq 0
    \end{align}

    We write this in tableau form: \\
    \begin{tabular}{c c c c c c c | c }
        \(x_1\) & \(x_2\) & \(x_3\) & \(x_4\) & \(w_1\) & \(w_2\) & \(w_3\) & RHS \\
        1 & -1 & 1 & -2 & 1 & 0 & 0 & 3 \\
        2 & 2 & -1 & -2 & 0 & 1 & 0 & 1 \\
        -1 & 1 & 3 & 3 & 0 & 0 & 1 & 6 \\
        \hline
        -2 & -2 & -3 & 1 & 0 & 0 & 0 & \(w-10\) \\
        3 & 1 & -1 & -1 & 0 & 0 & 0 & z
    \end{tabular}

    We perform the M.R.T., \(\frac{3}{1}\) vs \(\frac{1}{2}\), and chose \(\frac{1}{2}\). Thus, we pivot on \(x_1\): \\
    
    %\begin{tabular}
       % \(x_1\) & \(x_2\) & \(x_3\) & \(x_4\) & \(w_1\) & \(w_2\) & \(w_3\) & RHS \\
      %  0 & -2 & $\frac{3}{2}$ & -1 & 1 & $\frac{-1}{2}$ & 0 & $\frac{5}{2}$ \\
     %   2 & 2 & -1 & -2 & 0 & 1 & 0 & 1 \\
    %    -1 & 1 & 3 & 3 & 0 & 0 & 1 & 6 \\
   %     \hline
  %      -2 & -2 & -3 & 1 & 0 & 0 & 0 & \(w-10\) \\
 %       3 & 1 & -1 & -1 & 0 & 0 & 0 & z
 %   \end{tabular}

    \textbf{finish writing tableau for pivots}


    Since \(w=0\), the LP is feasible. However, \(z\) is not optimized because we 
    still have a negative coefficient in the objective function. Therefore, we pivot on (FINSIH THIS)
\end{answer}

\begin{definition}{\textbf{Multi-period Decision Models}}

    These are dynamic problems in where decisions must be 
    made over a series of time periods. Many of these problems can also be solved using \textbf{dynamic programming}\dots

    Idea of formulation: To formulate such problems as LPs, you must include linking constraints, which tie 
    together variables from one period to variables from previous periods.
\end{definition}

\begin{eg}{Staffing a Training Program}



\end{eg}
\begin{answer}
    We first chose our decision variables: \\
    \begin{align*}
        A_t&= \text{number of agents available at the beginning of each month t} \\
        H_t&= \text{number of students hired at beginning of month t}
    \end{align*}

    Then, we formulate an LP: \\
    \begin{align*}
        \text{min }350(H_1+\dots+h_6)+600(A_1+\dots+A_6) \\
        \text{s.t. }A_1&=60 \\
        150A_1-20H_1&\geq8000 \\
        150A_2-20H_2&\geq 9000 \\
        \vdots \\
        150A_6-20H_6&\geq11000 \\
    \end{align*}
\end{answer}
\textbf{FINISH WRITING THIS OUT}




\end{document}